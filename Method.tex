
\section{Method}
\subsection{The Fisher Matrix}
The elements of the Fisher matrix are defined as:
\begin{equation}
\mathcal{F}_{ij} = - \left \langle \frac{\partial^{2}ln{\mathcal{L}}}{\partial p^i\partial p^j } \right \rangle
\end{equation}
where $\mathcal{L}$ is the likelihood and $p^{i}$, $p^{j}$ are the $i^{th}$ and $j^{th}$ model parameters.
\\
The Fisher matrix is a powerful tool in experimental design, used for forecasting the upper limits on the precision of an experiment. This comes as a result of the Cram\`{e}r-Rao theorem.
\\
The Cram\`{e}r-Rao theorem states that the variance of an unbiased estimator, p is greater than or equal to the reciprocal of its Fisher information. The Cram\`{e}r-Rao bound can be expressed as:

\begin{equation}
var(p) \geq \frac{1}{\langle (\frac{\partial ln(\mathcal{L}(p \vert x))}{\partial p})^2 \rangle}
\end{equation}
Where $\langle (\frac{\partial ln(\mathcal{L}(p \vert x))}{\partial p})^2 \rangle$ is the Fisher information.

\subsection{Covariances}
\subsection{extended datasets}
\subsection{crosschecks}
\subsection{tests}
