\section{Discussion and Future Work}
\subsection{Discussion}

PMFs are a prime candidate for a weak seed magnetic field that, after amplification by galactic dynamos, could explain the origins of the weak large-scale magnetic fields that thread the cosmos. The most stringent tests of PMFs currently come from their impact on the polarised B-modes (odd parity) in the CMB. Recent CMB polarisation measurements with the Planck satellite have led to a 95\% CL upper limit of the PMF strength smoothed over 1 megaparsec as $B_{1Mpc} \leq$ 4.4nG \cite{Ade:2015cva}; a similar analysis using the POLARBEAR B-mode power spectrum measurement led to an upper limit of $B_{1Mpc} \leq$ 3.9nG \cite{Ade:2015cao}.

We expect the field to undergo rapid improvements due to the expected exponential growth in the size of CMB experiments over the next few years. New CMB experiments with an order of magnitude more detectors than POLARBEAR will begin taking data next year (e.g Adv.ACTpol \cite{Henderson:2015nzj}, Simons Array \cite{Suzuki:2015zzg} and SPT-3G \cite{Benson:2014qhw}). Following these experiments in a planned so-called "CMB stage-4" experiment with approximately 300$\times$ more detectors than POLARBEAR that received high priority in the last US P5 report \cite{p5}.
Although still in the design and planning phase, the tentative schedule for CMB stage-4 is in the early 2020s. While these new CMB experiments are being built primarily to search for inflationary gravitational waves and neutrino masses, they will also be potent probes of the faint signature of PMFs on the polarised CMB. In this work I have addressed how well these experiments will constrain PMFs, and have considered how the experimental parameters of CMB stage-4 missions might be tuned to yield the best measurement of PMFs.

Using simulated CMB power spectra with an effective 2.5nG PMF and mock covariance matrices for future CMB experiments, I was able to forecast the limits on $\sigma(B_{1Mpc})$ for stage-3 and stage-4 CMB experiments. With these forecasts I was able to test the impact on $\sigma(B_{1Mpc})$ from varying the experimental design variables, survey area, noise, beam width, $\ell_{knee}$, beam uncertainty and calibration error. I found that the optimal experimental design for detecting a PMF with a field strength of 2.5nG maximises the scan area, minimises the noise, $\ell_{knee}$ and the beam width and is independent of beam uncertainty and calibration error. The lowest variance I found with this method was $\sigma(B_{1Mpc}) \geq 0.0036$nG for a survey area of 28877 square degrees with all other experimental variables fixed. This result reflects the best case scenario for stage-4 CMB epxeriments. These results all drastically improve - by up to a factor of 100 - upon current-day constraints taken from Planck, which currently has $\sigma(B_{1Mpc}) \geq 0.38$nG.

In addition to finding the best experimental design variables I also plotted the 1$\sigma$ contours for current Planck data, stage-3 expected and stage-4 expected constraints on $B_{1Mpc}$, tensor-to-scalar ratio $r$, running of the scalar spectral index $n_{run}$ and the effective number of neutrinos $N_{eff}$. As expected, stage-3 and stage-4 experiments improve constraints for all these parameters. Stage-3 experiments give $\sigma(B_{1Mpc}) = 0.00627$, $\sigma(N_{eff}) = 0.00536$, $\sigma(r) = 0.00169$ and $\sigma(N_{eff}) = 0.00477$, an improvement over Planck by up to a factor of 100. There are no clear degeneracies between $n_{run}$ and $B_{1Mpc}$ nor $N_{eff}$ and $B_{1Mpc}$, however there is a degeneracy between $r$ and $B_{1Mpc}$, which is not broken by stage-3 or stage-4.

\subsection{Future Work}

There are still many avenues for further research in this area. In the realm of improving constraints and making predictions for future experiments we could make more joint constraints on model paramters for extensions to the standard cosmology with $B_{1Mpc}$ and check if other pre-existing degeneracies are broken by the next generation CMB experiments. Another approach would be to see how small we can make our constraints on $B_{1Mpc}$ and the model parameters already studied. By adding non-CMB data such as baryonic acoustic oscillations or supernovae observations, $\sigma(B_{1Mpc})$ will decrease further, but by how much? Adding data from other experiments that aim to measure PMFs such as cosmic shear surveys \cite{2012JCAP...11..055F} may serve to break the degeneracy between the tensor-to-scalar ratio and the PMF strength in our data.

Another option is to change our assumptions about our detection of PMFs. In this analysis we assumed that future experiments would detect a PMF with $B_{1Mpc} =$ 2.5nG and $A_{PMF} =$ 1 however a PMF may not have these values. In this limit, $\sigma(B_{1Mpc})$ scales linearly with $\sigma(A_{PMF})$, but as we move away this relationship doesn't hold. For a different strength PMF our forecasts will be very different. A more prudent approach would be to assume that there is no PMF by setting both $B_{1Mpc}$ and $A_{PMF}$ to zero in order to forecast for the minimum field strength we could discern with the new CMB experiments.

This analysis uses only the 2-point estimator to detect PMFs. We could also try forecasting for higher moments. POLARBEAR tried both 2-point estimation and 4-point estimation \cite{Ade:2015cao}, however did not see any major improvements using 4-point estimation. Due to the lower noise levels of future CMB experiments, we should expect to see a dramatic improvement of 4-point estimators, such that they provide constraints that at least rival 2-point estimators.

If PMFs are detected, we could also forecast for other PMF parameters. Two options with lots of promise are $n_{PMF}$, the spectral index for the PMF power spectrum and $\beta$, the PMF era. $n_{PMF}$ is a direct probe of the generation mechanism for PMFs. By tightening the confidence intervals on $n_{PMF}$ we can more confidently rule out proposed generation mechanisms. $\beta$ on the other hand, tell us the age of the Universe when PMFs first formed. If we can measure these two parameters we would have a very clear picture of how PMFs fit into the cosmos. To do this though, we must first design an experiment that can detect these parameters, hence it is essential that we repeat the analysis done here to find the optimal experimental designs for measuring $n_{PMF}$ and $\beta$.
