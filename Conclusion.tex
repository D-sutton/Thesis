\section{Discussion and Future Work}
\subsection{Discussion}

By combining simulated CMB power spectra with an assumed PMF field strength of 2.5nG smoothed over 1Mpc with covariance matrices from mock stage-3 and stage-4 CMB experiments, I was able to produce covariance matrices over $\Lambda CDM$ model parameters, extended model parameters and the PMF strength. 

\subsection{Future Work}

There are still many avenues for further research in this area including repeating the analysis for other PMF parameters, tightening the constraints with extra data and adding more $\Lamda CDM$ extension parameters to the Fisher matrix.

In addition to forecasting $\sigma(B_{1Mpc})$, we could also forecast for other PMF parameters. Two options with lots of promise are $n_{PMF}$, the spectral index for the PMF power spectrum and $\beta$, the PMF era. $n_{PMF}$ is a direct probe of the generation mechanism for PMFs. By tightening the confidence intervals on $n_{PMF}$ we can more confidently rule out proposed generation mechanisms. $\beta$ on the other hand, tell us the age of the Universe when PMFs first formed. If we can measure these two parameters we would have a very clear picture of how PMFs fit into the cosmos. To do this though, we must first design an experiment that can detect these parameters, hence it is essential that we repeat the analysis done here to find the optimal experimental designs for measuring $n_{PMF}$ and $\beta$.

Another direction we could take is to further improve upon the forecasted constraints here by using covariances from other experiments, inverting them into a Fisher matrix and adding them to our own. By doing this we may not only find that $\sigma(B_{1Mpc})$ decreases significantly but also break degeneracies between PMF parameters and the extensions to $\Lambda CDM$. In a similar vein it may prove useful to add further extensions to $\Lambda CDM$ model parameters to find out how a PMF analysis would constrain the wider parameter space.

Finally, in this analysis we assumed that future experiments would detect a PMF with a field strength of 2.5nG. This is not necessarily the case, POLARBEAR has constrained the field strength to $B_{1Mpc} \leq 3.9nG$, so all values in this domain are plausible. In the future we could carry out an analysis for a range of different magnetic field strengths.



