\section{Discussion and Future Work}
\subsection{Discussion}

PMFs are a prime candidate for a weak seed magnetic field that, through amplification by galactic dynamos could explain the origins of the weak large-scale magnetic fields that thread the cosmos. Indirect detection of PMFs through measuring the B-mode polarisation power spectrum in the CMB holds promise for testing the model. Previous work from POLARBEAR has given the upper limit on the primordial magnetic field strength smoothed over 1 megaparsec as $B_{1Mpc} \leq 3.9nG$. In late 2016 the next generation of CMB experiments, known as stage-3 will begin operation followed by stage-4 in the early 2020s. Though not the main science goal of these experiments, stage-3 and stage-4 may be able to detect the faint traces of PMFs within the CMB. In the interest of time and money, it is useful to know beforehand just how sensitive these experiments will be to PMF signals.

Using simulated CMB power spectra with an effective 2.5nG PMF and mock covariance matrices for future CMB experiments, I was able to forecast the limits on $\sigma(B_{1Mpc})$ for stage-3 and stage-4 CMB experiments. With these forecasts I was able to test the impact on $\sigma(B_{1Mpc})$ from varying the experimental design variables, survey area, noise, beam width, $\ell_{knee}$, beam uncertainty and calibration error. I found that the optimal experimental design for detecting a PMF with a field strength of 2.5nG maximises the scan area, minimises the noise, $\ell_{knee}$ and the beam width and is independent of beam uncertainty and calibration error. The lowest variance I found with this method was $\sigma(B_{1Mpc}) \geq 0.0036nG$ for a survey area of 28877 square degrees with all other experimental variables fixed. This result reflects the best case scenario for stage-4 CMB epxeriments. These results all drastically improve - by up to a factor of 100 - upon current-day constraints taken from PLANCK, which currently has $\sigma(B_{1Mpc}) \geq 0.38nG$.

In addition to finding the best experimental design variables I also plotted the 1$\sigma$ contours for current PLANCK data, stage-3 expected and stage-4 expected constraints on $B_{1Mpc}$, tensor-to-scalar ratio $r$, running of the scalar spectral index $n_{run}$ and the effective number of neutrinos $N_{eff}$. As expected, stage-3 and stage-4 experiments improve constraints for all these parameters. We see that future CMB experiments offer moderate improvements on $n_{run}$ and major improvements on $N_{eff}$ and $r$ from current PLANCK results. There are no clear degeneracies between $n_{run}$ and $B_{1Mpc}$ nor $N_{eff}$ and $B_{1Mpc}$, however there is a degeneracy between $r$ and $B_{1Mpc}$, which is not broken by stage-3 or stage-4.

\pagebreak

\subsection{Future Work}

There are still many avenues for further research in this area including repeating the analysis for other PMF parameters, tightening the constraints with extra data and adding more $\Lamda CDM$ extension parameters to the Fisher matrix.

In addition to forecasting $\sigma(B_{1Mpc})$, we could also forecast for other PMF parameters. Two options with lots of promise are $n_{PMF}$, the spectral index for the PMF power spectrum and $\beta$, the PMF era. $n_{PMF}$ is a direct probe of the generation mechanism for PMFs. By tightening the confidence intervals on $n_{PMF}$ we can more confidently rule out proposed generation mechanisms. $\beta$ on the other hand, tell us the age of the Universe when PMFs first formed. If we can measure these two parameters we would have a very clear picture of how PMFs fit into the cosmos. To do this though, we must first design an experiment that can detect these parameters, hence it is essential that we repeat the analysis done here to find the optimal experimental designs for measuring $n_{PMF}$ and $\beta$.

Another direction we could take is to further improve upon the forecasted constraints here by using covariances from other experiments, inverting them into a Fisher matrix and adding them to our own. By doing this we may not only find that $\sigma(B_{1Mpc})$ decreases significantly but also break degeneracies between PMF parameters and the extensions to $\Lambda CDM$. In a similar vein it may prove useful to add further extensions to $\Lambda CDM$ model parameters to find out how a PMF analysis would constrain the wider parameter space.

Finally, in this analysis we assumed that future experiments would detect a PMF with a field strength of 2.5nG. This is not necessarily the case, POLARBEAR has constrained the field strength to $B_{1Mpc} \leq 3.9nG$, so all values in this domain are plausible. In the future we could carry out an analysis for a range of different magnetic field strengths.



