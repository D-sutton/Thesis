\section{Introduction}

The cosmic microwave background (CMB) is the oldest light in the Universe. It was formed around 300,000 years after the Big Bang, during the era of recombination. During this time the Universe was still hot and dense, emitting high energy photons through black body radiation. However, the Universe had cooled sufficiently to allow electrons and protons to form neutral hydrogen, making space transparent to light. The photons from this era have since been cosmologically redshifted into the microwave frequency band. The temperature of the CMB photons is ~2.7K, which is currently the equilibrium temperature of the Universe. 
\\\\
As a relic of the early Universe, the CMB is a powerful tool for probing high energy physics, piecing together the history of the Universe and mapping out large scale strucutres. The discovery of the CMB was a landmark success for verifying the Big Bang Theory and is still studied to test $\Lambda$CDM Cosmology. Since its discovery in 1964, the CMB has been studied extensively. In 1989 COBE launched and found the CMB matches the blackbody spectrum with a temperature of 2.725 $\pm$ 0.002K as well as detecting anisotropies to one part in 100,000. In 2001 WMAP began its 9 year run, solidifying $\Lambda$CDM parameters and beginning the era of precision cosmology. In

\iffalse
> what the cmb is 
	- Oldest light in the Universe
	- from the era of recombination
	- in the microwave frequency band
	- ~2.7K, residual temperature of the Universe
> what is its significance
	- evidence for the big bang model
	- can large scale structure (lensing)
	- piece together history of Universe - pmfs and PGW - B-modes
	- not much else tbh???
> previous work done on the cmb
	- COBE
	- WMAP
	- PLANCK
	- etc.
\fi
\subsection{Large Scale Magnetic Fields}

Galaxies, Galaxy clusters and the Intercluster medium all exhibit weak magnetic fields on the order of microgauss coherent over kiloparsecs or megaparsecs

\iffalse
> What?
	- weak magnetic fields coherent over kilo to megaparsec scales
	- evidence: Faraday rotation + synchrotron emission + zeeman effect
> How?
	- hydrodynamics
	- stars
	- pmfs
\fi
\subsection{Primordial Magnetic Fields}
	- seed magnetic field produced in the early universe
	- undetected
	- predicted magnitude of < 1 nG.
	- evolves through magnetohydrodynamical processes

\subsection{CMB Polarisation}
	< 'borrow' from lit review? >
	
\subsection{CMB-S3}
	- stage 3 of CMB experiments operating from 2016 to 2020.
	- n detectors, m detector years
	- primary science goal is to detect PGW from cosmic inflation through B-modes.
	- could also make a detection of PMFs.
\subsubsection{SPT-3G}
	- South Pole Telescope
	- n detectors
	- starting/fishishing dates
	- area
	- etc.
\subsubsection{Advanced ACTPol}
\subsubsection{The Simons Array}
\subsection{CMB-S4}
	- stage 4 CMB experiments still in planning stages - nothing concrete yet.
	- n detectors, m detector years
	- running through the 2020s