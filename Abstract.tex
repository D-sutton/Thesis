\section*{Abstract}

Weak large-scale magnetic fields coherent over megaparsecs permeate galaxies and galaxy clusters. Primordial magnetic fields are proposed as a seed magnetic field produced in the early Universe giving rise to these fields observed today. Recently the field strength of primordial magnetic fields have been constrained by measuring the B-mode power spectrum in the cosmic microwave background (CMB). Stage-3 CMB experiments such as SPT-3G and The Simons Array entering operation in 2016 will be able to provide more precise constraints on the primordial magnetic field strength. We forecast the minimum uncertainty on the primordial magnetic field strength, $\sigma(B_{1Mpc})$ for upcoming stage-3 and planned stage-4 CMB experiments by constructing the Fisher matrix for these experiments. We find the minimum uncertainty to be $\sigma(B_{1Mpc}) \geq 0.0035nG$ for a late stage-3 CMB experiment. We show the optimum CMB experiment for detecting primordial magnetic fields minimises noise, $\ell_{knee}$ and beam width.
Finally we constrain the running of the scalar index $n_{s}$, the effective neutrino number $N_{eff}$ and the tensor-to-scalar ratio $r$ against $B_{1Mpc}$ and find that stage-3 and stage-4 CMB experiments improve constraints on these extended model parameters compared to current PLANCK constraints but do not break the degeneracy between $B_{1Mpc}$ and $r$.