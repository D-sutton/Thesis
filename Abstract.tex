\section*{Abstract}

The origin of the weak large-scale magnetic fields coherent over megaparsecs that permeate galaxies and galaxy clusters is a mystery. 
One proposed source is a primordial magnetic field (PMF) that is then magnified to give rise to the fields observed today. The best upper limit we have on PMFs come from looking for the Faraday rotation they would induce on maps of the polarisation of the cosmic microwave background (CMB). New CMB experiments such as SPT-3G and The Simons Array entering operation in 2017 should be able to susbstantially improve the constraints on the primordial magnetic field strength. We forecast the minimum uncertainty on the primordial magnetic field strength, $\sigma(B_{1Mpc})$ for these experiments, and an even larger experiment proposed for 2021.
Using a Fisher matrix formalism we find the minimum uncertainty predicted for an experiment like the Simons Array to be $\sigma(B_{1Mpc}) \geq 0.0035$nG, a factor 100 better than current PLANCK constraints. We show the optimum CMB experiment for detecting primordial magnetic fields minimises noise, $\ell_{knee}$ and beam width. Finally we constrain the running of the scalar index $n_{s}$, the effective neutrino number $N_{eff}$ and the tensor-to-scalar ratio $r$ against $B_{1Mpc}$ and find that these future CMB experiments improve constraints on these extended model parameters compared to current PLANCK constraints but do not break the degeneracy between $B_{1Mpc}$ and $r$. The presented results can be used to guide the design of future studies of PMFs.