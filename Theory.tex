\section{PMF Theory}

\iffalse
Observations of Faraday rotation in the polarisation of radio signals reveal the existence of weak coherent magnetic fields over large scales. Magnetic fields have strengths of order microgauss coherent over kPc scales. Similarly the intracluster medium possesses microgauss fields over tens of kPc. In addition, the voids between galaxy clusters are believed to have coherent nanogauss fields.
\\\\
The origin of these fields is not explained in $\Lambda$CDM cosmology. [why]. A solution to this problem are seed magnetic fields produced early in the Universe. These seed fields would be amplified through galactic dynamos and magnetohydrodynamics into the magnetic fields we observe in the present day.
\\\\
Primordial Magnetic Fields (PMFs) are a candidate for the seed magnetic fields. PMFs would have been produced during the early Universe. They may have been a result of conformal symmetry breaking during inflation, or during a phase transition - QCD to weak, say. The Biermann battery is the most standard mechanism for PMF production.
\fi
\subsection{Biermann Batteries}

The large-scale magnetic fields we see need to have had some initial seed field, but of course this raises the question: Where did the seed field come from? The most popular model for seed magnetic field generation from zero initial conditions is the 'Biermann battery' proposed by Biermann in 1950. Biermann batteries form in highly ionised environments such as the plasma shortly afer the Big Bang. Within the plasma, ions are drawn to regions of lower density and lower temperature. Since the the constituents of the plasma - protons and electrons - have different masses they flow at different rates resulting in a net flow of charge. If this flow of current forms a loop, then by Faraday's law of induction, a magnetic field is produced by the battery.

The magnetic field produced by the Biermann battery is described by:
\begin{equation}
\frac{\partial \vec{B}}{\partial t} = \nabla\times(\vec{U}\times\vec{B}-\eta\nabla \times\vec{B}) - \frac{c k_{b}}{e}\frac{\nabla n_e}{n_e} \times \nabla T
\end{equation}
\\
The final term, $\nabla n_e \times \nabla T$ is the source term describing the Biermann battery effect. In order for this term to be non-zero and hence to have a Biermann battery, gradients of the electron density and the temperature must be non-parallel.
\iffalse
\\
\begin{figure}[h]
\centering
\includegraphics[scale=1]{}
\caption{}
\end{figure}
\\
\fi
\subsection{Other Methods of PMF Generation}

\subsubsection*{Inflation}
Cosmic inflation is an attactive model for PMF generation. During inflation the Universe isn't yet an ionised plasma. This means that the Universe is not a good conductor. In this state magnetic flux need not be conserved so it's possible for a seed field to emerge spontaneously.

The rapid expansion of the Universe in this epoch stretches out modes. Inflation stretches out quantum fluctuations into large scale density perturbations which seed the structure of the Universe. Similarly, inflation could stretch out small, weak magnetic fields to megaparsec scales as required.

A problem for this model is that it requires inflation to break conformal symmetry to produce this weak seed field initially. Turner and Widrow (1988) present one such model for conformal symmetry breaking leading to PMF generation in \cite{PhysRevD.37.2743}.
\\
\subsubsection*{Phase Transitions}
PMFs may have also been produced by early phase transitions, such as the QCD transition or the electroweak phase transition. During a phase transition bubbles of the new phase form within the previous phase, these bubbles grow and collide until the entire Universe reaches the lower phase. These phase transitions bring on non-equilibrium processes such as leptogenesis and baryogenesis, which may be responsible for producing some weak magnetic fields. Within a phase transition, a collision between bubbles will produce turbulence leading to dynamos which will serve to spin up the magnetic fields into the strengths required to match the field strengths observed today.

\subsection{Effect of PMFs on the Cosmic Microwave Background}
If PMFs have a field strength ~1nG then their signatures will be detectable in the CMB B-mode polarisation power spectrum. Just as extragalactic magnetic fields Farday rotate radio and X-ray signals, PMFs would induce Faraday rotation within CMB polarisation. The net effect is that a fraction of E-mode polarisation would be transformed into B-mode polarisation. 
The PMF power spectrum is given by:

\begin{equation}
P(k) = A_{PMF}k^{n_B}
\end{equation}

Where $A_{PMF}$ is the PMF amplitude and $n_{B}$ is the PMF spectral index. Since the scale of the PMF power spectrum will depend on the age of the Universe when they first formed, the spectral index is sensitive to the mechanism that first produced PMFs. If $n_{B}$...
In order to measure the strength of PMFs we focus our attention to the amplitude, $A_{PMF}$. The PMF amplitude is related to $B_{1Mpc}$, the strength of PMFs coherent over 1 megaparsec by the following relation:
\begin{equation}
A_{PMF} \propto {B_{1Mpc}}^4
\end{equation}

\\
\begin{figure}[h]
\centering
\includegraphics[scale=0.7]{images/PMFpower.png} 
\caption{Plot of the CMB power spectrum. The dotted green and yellow lines show the power spectrum for vector and tensor PMFs. Vector PMFs have more power on smaller scales (larger $\ell$) and tensor PMFs have more power on larger scales (smaller $\ell$). The purple line shows the expected combined PMF plus lensing effects on the CMB. Compared to the blue dotted line for lensing alone, a PMF-influenced CMB power spectrum may be detectable on smaller scales than on larger scales.}
\end{figure}
\\

Recent work from PLANCK (2015) has constrained the primordial magnetic field strength coherent over 1 Mpc to $B_{1Mpc}$ $<$ 4.4nG \cite{Ade:2015cva}. In 2016 POLARBEAR modestly improved this constraint to $B_{1Mpc}$ $<$ 3.9nG \cite{Ade:2015cao}.
\subsection{Other effects of PMFs}

In addition to CMB polarisation, PMFs will affect the have effects on large scale structure and Big Bang Nucleosynthesis (BBN).

\subsubsection*{Large Scale Structure}

PMFs can indirectly shape the the structure of the cosmos. At early times a PMF can exert a Lorentz force on baryonic matter. Since baryonic matter interacts gravitationally with dark matter, it follows that PMFs have an influence on matter distribution. This effect can be observed in the density perturbation amplitude over 8 Mpc, $\sigma_8$. One can also observe its effects by looking for changes in the matter power spectrum. Figure X shows the impact of PMFs on the matter power spectrum for Universes with either massless or massive neutrinos.
\\
\begin{figure}[h]
\centering
\includegraphics[scale=1.1]{images/matterpower.png} 
\caption{Plot of the matter power spectrum. Here $B_{\lambda}$ is equal to $B_{1Mpc}$ The left panel gives the power spectrum in the range $0.01 Mpc^{-1} < k/h < 0.2 Mpc^{-1}$. At small scales (small k/h) all models fall within the experimental error, however the Universe without massive neutrinos or PMFs has more power at these scales. The right panel gives the power spectrum in the range $0.1 Mpc^{-1} < k/h < 0.2 Mpc^{-1}$. At larger scales, the matter power spectrum is significantly affected by the presence and strength of PMFs.}
\end{figure}
\\

So far, there are no constraints on the PMF amplitude from large scale structure as the interactions happen in the non-linear regime, making measurements of its effects diffucult at the current time.


\subsubsection*{Big Bang Nucleosynthesis}

If PMFs existed prior to BBN then we expect to see its signature in nuclear abundances. Primordial magnetic fields contribute to the overall energy density of the Universe and therefore vary the rate of expansion. If the rates of expansion in the Universe differ then so does the rate at which the Universe cools. If PMFs were indeed formed prior to BBN then the freeze-out times of reactions during BBN would differ from standard model values and hence so would primordial abundances of Helium and Hydrogen.

The strongest constraints from BBN have $B_{1Mpc}$ $<$ 1.5 $\mu G$ \cite{PhysRevD.86.063003}. These constraints are far weaker than those currently given by CMB polarisation measurements.

\iffalse Weak constraints, Yamazaki in 2012 cites B_1mpc < 1.5 $\mu G$, which isn't very telling.
\fi

\subsection{Other Sources of Cosmic Birefringence}
The rotation of E-mode polarisation to B-mode polarisation is not a phenomenon unique to PMFs. Another mechanism for this effect may be quintessence. Quintessence is an alternative explanation to the cosmological constant for the accelerating expansion of the Universe. It argues that there may exist a long-range pseudoscalar field that can very weakly couple to baryons. The interaction is described by the Chern-Simmons term:
\begin{equation}
\mathcal{L} \propto \frac{\phi}{2M}F_{\mu \nu}\tilde{F}^{\mu \nu}
\end{equation}
\\\\
Where $\phi$ is the pseudoscalar field and M is the mass of the field boson. If photons couple to this field, then their polarisation will be rotated, just as they would if there were a PMF. The rotation angle due to the pseudoscalar field is given by:
\begin{equation}
\alpha = \frac{1}{M} \int{d\eta \dot{\phi}}
\end{equation}
Where $\dot{\phi}$ is integrated over the conformal time $\eta$

Currently we constrain the effects of cosmic birefringence with an equivalent effective PMF, however by comparing the two-point and four-point correlation functions it is possible to differentiate between effects due to PMFs and effects due to conjectured quintessence models. \cite{Ade:2015cao}




