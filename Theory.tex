\section{PMF Theory}

\iffalse
Observations of Faraday rotation in the polarisation of radio signals reveal the existence of weak coherent magnetic fields over large scales. Magnetic fields have strengths of order microgauss coherent over kPc scales. Similarly the intracluster medium possesses microgauss fields over tens of kPc. In addition, the voids between galaxy clusters are believed to have coherent nanogauss fields.
\\\\
The origin of these fields is not explained in $\Lambda$CDM cosmology. [why]. A solution to this problem are seed magnetic fields produced early in the Universe. These seed fields would be amplified through galactic dynamos and magnetohydrodynamics into the magnetic fields we observe in the present day.
\\\\
Primordial Magnetic Fields (PMFs) are a candidate for the seed magnetic fields. PMFs would have been produced during the early Universe. They may have been a result of conformal symmetry breaking during inflation, or during a phase transition - QCD to weak, say. The Biermann battery is the most standard mechanism for PMF production.
\fi
\subsection{Biermann Batteries}

The large-scale magnetic fields we see need to have had some initial seed field, but of course this raises the question: Where did the seed field come from? The most popular model for seed magnetic field generation from zero initial conditions is the 'Biermann battery' proposed by Biermann in 1950. Biermann batteries form in highly ionised environments such as the plasma shortly afer the Big Bang. Within the plasma, ions are drawn to regions of lower density and lower temperature. Since the the constituents of the plasma - protons and electrons - have different masses they flow at different rates resulting in a net flow of charge. If this flow of current forms a loop, then by Faraday's law of induction, a magnetic field is produced by the battery.

The magnetic field produced by the Biermann battery is described by:
\begin{equation}
\frac{\partial \vec{B}}{\partial t} = \nabla\times(\vec{U}\times\vec{B}-\eta\nabla \times\vec{B}) - \frac{c k_{b}}{e}\frac{\nabla n_e}{n_e} \times \nabla T
\end{equation}
\\
The final term, $\nabla n_e \times \nabla T$ is the source term describing the Biermann battery effect. In order for this term to be non-zero and hence to have a Biermann battery, gradients of the electron density and the temperature must be non-parallel.

Biermann batteries are also predicted to occur at later times in the Universe




\subsection{Other Methods of PMF Generation}
\subsection{Effect of PMFs on the Cosmic Microwave Background}
If PMFs have a field strength ~1nG then their signatures will be detectable in the CMB B-mode polarisation power spectrum. Just as extragalactic magnetic fields Farday rotate radio and X-ray signals, PMFs would induce Faraday rotation on CMB polarisation. The net effect is that a fraction of E-mode polarisation would be transformed into B-mode polarisation.
\subsection{Other effects of PMFs}
\subsection{Cosmic Birefringence}
Measuring a conversion of E-mode power to B-mode power consistent with expected PMF effects doesn't strictly confirm the existence of PMFs. PMFs are not the only mechanism that could rotate the polarisation of 



